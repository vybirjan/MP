\chapter{Úvod}

\section{Motivace}

Důvodem vzniku této práce byla potřeba přidat do existujích aplikací napsaných
v jazyce Java na platformě Eclipse RCP podporu pro licencování.

Výrazem licence se v kontextu vývoje software myslí oprávnění, kterým autor
software umožňuje jeho použití dalším osobám. Při komerční distribuci software
se často vyskytuje potřeba upravit funkčnost na základě zakoupené licence.
Takovéto úpravy mohou zahrnovat například zpřístupnění nebo zakázání některých
vlastností produktu, omezení možností opakované instalace a kopírování případně
časové omezení určité funkcionality.

Vhodný nástroj pro licencování umožňí 

 
	
\section{Cíle práce}
Cílem práce je prozkoumat dostupná řešení, která umožňjí přidat do aplikací
podporu pro licencování, a navrhnou vlastní řešení splňující následující
požadavky:
\begin{itemize}
  \item Implementace v jazyce Java
  \item Podpora pro aplikace napsané na platformě Eclipse RCP
  \item Funkčnost na operačních systémech Windows a Linux
  \item Snadná integrovatelnost do již existujících aplikací
\end{itemize}

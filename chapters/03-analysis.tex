\chapter{Analýza}

V této kapitole je provedena analýza požadavků a je navržena architektura
výsledné aplikace.

\section{Požadavky}

\subsection{Nefunkční požadavky}

Výsledný navrhovaný nástroj musí splňovat následující podmínky

\subsection{Implementace v jazyce Java}

Vzhledem k tomu, že důvodem ke vzniku práce bylo dodání licencovacího mechanismu
do existujících aplikací napsaných v jazyce Java, jedním z hlavních požadavků
je celý navržený systém implementovat v Javě. Ačkoliv Java umožňuje využívat
knihovny napsané například v C/C++ pomocí JNI\cite{jni} nebo JNA\cite{jna}, je
použití nativních knihoven mnohem snadnější.


\subsection{Podpora pro aplikace napsané na platformě Eclipse RCP}

Důvodem tohoto požadavku je fakt, že existující aplikace jsou napsány právě na
platformě Eclipse RCP\cite{eclipse-rcp}. Pro snadnější integraci by měla být
klientská část navrhovaného systému distribuovaná v podobě pluginu.

Důsledkem tohoto požadavku je kromě podoby výsledné distribuce klientské části
také omezení týkající se knihoven třetích stran. Využívany by měli být především
takové knihovny, které jsou obvykle obsaženy ve většině distribucí aplikací
napsaných na platformě Eclipse RCP. Například v případě použití GUI by měla být
použita knihovna SWT\cite{swt}.

\subsection{Funkčnost na operačních systémech Windows a Linux}

Tento požadavek opět souvisí s tím, že existující aplikace podporují právě tyto
operační systémy. Navíc jsou systémy Windows a Linux nejsnáze dostupné pro
testovací účely. 

Ačkoliv je Java multiplatformní jazyk, možnosti týkající se zístkávání informací
specifických pro konkrétní operační systém jsou značně omezené. Pokud chceme
vázat licenci na konkrétní hardware, je zapotřebí mnohem těsnější vazby s
operačním systémem, než jakou Java umožňuje. Z tohoto důvodu bude nutné část
funkcionality přizpůsobit konkrétnímu operačnímu systému a tuto část
implementovat pro všechny podporované platformy. Jako cílové byly zvoleny
operační systémy Windows a Linux, mělo by ale být možné v případě potřeby snadno
rozšířit podporu i pro další operační systémy.


\subsection{Snadná integrovatelnost do již existujících aplikací}
 
 
  
% \item Centrální správa licencí
 % \item Webové GUI pro snadnějš správu licencí
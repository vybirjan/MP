\chapter{Analýza}

V této kapitole je provedena analýza požadavků a je navržena architektura
výsledné aplikace.



\section{Požadavky}

\subsection{Nefunkční požadavky}

Výsledný navrhovaný nástroj musí splňovat následující podmínky

\subsubsection*{Implementace v jazyce Java}

Vzhledem k tomu, že důvodem ke vzniku práce bylo dodání licencovacího mechanismu
do existujících aplikací napsaných v jazyce Java, jedním z hlavních požadavků
je celý navržený systém implementovat v Javě. Ačkoliv Java umožňuje využívat
knihovny napsané například v C/C++ pomocí JNI\cite{jni} nebo JNA\cite{jna}, je
použití nativních knihoven mnohem snadnější.


\subsubsection*{Podpora pro aplikace napsané na platformě Eclipse RCP}

Důvodem tohoto požadavku je fakt, že existující aplikace jsou napsány právě na
platformě Eclipse RCP\cite{eclipse-rcp}. Pro snadnější integraci by měla být
klientská část navrhovaného systému distribuovaná v podobě pluginu.

Důsledkem tohoto požadavku je kromě podoby výsledné distribuce klientské části
také omezení týkající se knihoven třetích stran. Využívany by měli být především
takové knihovny, které jsou obvykle obsaženy ve většině distribucí aplikací
napsaných na platformě Eclipse RCP. Například v případě použití GUI by měla být
použita knihovna SWT\cite{swt}.

\subsubsection*{Funkčnost na operačních systémech Windows a Linux}

Tento požadavek opět souvisí s tím, že existující aplikace podporují právě tyto
operační systémy. Navíc jsou systémy Windows a Linux nejsnáze dostupné pro
testovací účely. 

Ačkoliv je Java multiplatformní jazyk, možnosti týkající se zístkávání informací
specifických pro konkrétní operační systém jsou značně omezené. Pokud chceme
vázat licenci na konkrétní hardware, je zapotřebí mnohem těsnější vazby s
operačním systémem, než jakou Java umožňuje. Z tohoto důvodu bude nutné část
funkcionality přizpůsobit konkrétnímu operačnímu systému a tuto část
implementovat pro všechny podporované platformy. Jako cílové byly zvoleny
operační systémy Windows a Linux, mělo by ale být možné v případě potřeby snadno
rozšířit podporu i pro další operační systémy.


\subsubsection*{Snadná integrovatelnost do již existujících aplikací}
 
K rozhodnutí o přidání funkčnosti pro licencování do aplikace může dojít až v
pozdější fázi vývoje, případně může být požadavek licencování přidat do již
hotového produktu. Z tohoto důvodu by měla být integrace co nejjednodušší.
Modulární architektura platformy Eclipse RCP toto v případě integrace do
desktopové aplikace usnadní. Stejný požadavek ale můžemem mít také na serverové
straně. Pokud již máme existující databázi uživatelů, mohli bychom ji chtít
využít pro vystavení licencí.
  
\subsubsection*{Centrální správa licencí}

Aby bylo možné mít přehled o všech vydaných licencích, je nutné licence
spravovat centrálně.

\subsubsection*{Webové GUI pro snadnějš správu licencí}

Protože o vydávání licencí se běžně stará obchodní oddělení, je potřeba
vytvořit grafické uživatelské rozhraní, které umožní uživatelům spravovat vydané
licence. Z hlediska snadnosti použití a platformní nezávislosti se jako
nejvhodnější jeví webové rozhraní.


\subsection{Funkční požadavky}

\subsubsection*{Vlastnosti licence}

Licence vydávané a ověřované implementovaným licenčním nástrojem musí mít
následující vlastnosti.

\begin{itemize}
  \item Časové omezení – počátek nebo konec platnosti licence je možné časově
  omezit
  \item Vazba na hardware – platnost licence je možné vázat na konkrétní
  hardware
  \item Licencované vlastnosti - do vystavené licence je možné zanást
  informace o povolených vlastnostech licencovaného produktu.
\end{itemize}

\subsubsection*{Vlastnosti klientské části}

Tyto vlastnosti bude mít klientská část licenčního mechanismu, která se bude
zabudovávat do aplikace.

\begin{itemize}
  \item Online kontrola platnosti – platnost licence bude je ověřit proti
  licenčnímu serveru.
  \item Offline kontrola platnosti – platnost údajů uvedených v licenci je
  možné ověřit bez nutnosti připojení k centrálnímu serveru. Jedná se především
  o kontrolu dat počátku a konce platnosti a kontroly vazby na hardware.
\end{itemize}

\subsubsection*{Vlastnosti serverové části}

\begin{itemize}
  \item Zobrazení přehledu vystavených licencí
  \item Vytvoření nové licence
  \item Zneplatnění licence
  \item Ověření platnosti vydané licence
\end{itemize}

\section{Architektura}

Jako nejvhodnější architektura pro navrhovaný produkt je typ klient-server.
Rozvržení jednotlivých komponent je zobrazeno v diagramu na obrázku
\ref{fig:deployment}.

\begin{figure}[H]
\begin{center}
\includegraphics[width=12cm]{figures/deployment.png}
\caption{Diagram nasazení jednotlivých komponent}
\label{fig:deployment} 
\end{center}
\end{figure}

\subsection{Serverová část}

Účelem serverové části je poskytovat informace o vydaných licencích a ověřovat
jejich platnost. Serverová část může být integrována do existující serverové
aplikace, nebo nasazena jako samostatná aplikace. Komponenty serverové aplikace
jsou na obrázku \ref{fig:components-server}.


\begin{figure}[H]
\begin{center}
\includegraphics[width=12cm]{figures/components-server.png}
\caption{Diagram komponent serverové části}
\label{fig:components-server} 
\end{center}
\end{figure}

Pro zajištění rozšiřitelnosti, snadné integrace do existujících aplikací a dobré
testovatelnosti je část funkcionality serverové části rozdělena do několika
komponent s definovaným rozhraním.

\subsubsection*{License Server}

Hlavní část serverové aplikace obsahující logiku spojenou s ověřováním
licencí. Jejím úkolem je odpovídat na požadavky od klientských aplikací na
ověření platnosti licence uložené v databázi. Jendotlivé části zajišťující
komunikaci s klienty a databází jsou dekomponovány do samostatných částí a
komunikace s nimi probíhá pomocí pevně daného rozhraní. To umožňuje udržet
komponentu nezávislou na použitém komunikačním protokolu a databázové vrstvě.

\subsubsection*{Communication Service}

Communication Service je komponenta odpovědná za komunikaci s klientskými
aplikacemi. Jejím vyčleněním do samostatné části je dosaženo odstranění
závislosti na použitém protokolu pro komunikaci s klientskými aplikacemi.
Komponenta Communication Service překládá požadavky na ověření platnosti licence
z podoby specifické pro použitý komunikační protokol (XML, JSON, Serialozované
objekty, \ldots) do univerzální podoby akceptované částí License Server.

V implementační části bude tato komponenta naimplementována v podobě
RESTful\cite{rest} webové služby.

\subsubsection*{Data Source}

Komponenta Data Source slouží k načítání a ukládání informací o vydaných
licencích do persistentního úložiště. Poskytuje serverové aplikaci základní operace jako
nalezení licence podle identifikátoru a zaznamenání aktivace licence.  

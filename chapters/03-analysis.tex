\chapter{Analýza}

V této kapitole je provedena analýza požadavků a je navržena architektura
výsledné aplikace.



\section{Požadavky}

\subsection{Nefunkční požadavky}

Výsledný navrhovaný nástroj musí splňovat následující podmínky

\subsubsection*{Implementace v jazyce Java}

Vzhledem k tomu, že důvodem ke vzniku práce bylo dodání licencovacího mechanismu
do existujících aplikací napsaných v jazyce Java, jedním z hlavních požadavků
je celý navržený systém implementovat v Javě. Ačkoliv Java umožňuje využívat
knihovny napsané například v C/C++ pomocí JNI\cite{jni} nebo JNA\cite{jna}, je
použití nativních knihoven mnohem snadnější.


\subsubsection*{Podpora pro aplikace napsané na platformě Eclipse RCP}

Důvodem tohoto požadavku je fakt, že existující aplikace jsou napsány právě na
platformě Eclipse RCP\cite{eclipse-rcp}. Pro snadnější integraci by měla být
klientská část navrhovaného systému distribuovaná v podobě pluginu.

Důsledkem tohoto požadavku je kromě podoby výsledné distribuce klientské části
také omezení týkající se knihoven třetích stran. Využívany by měli být především
takové knihovny, které jsou obvykle obsaženy ve většině distribucí aplikací
napsaných na platformě Eclipse RCP. Například v případě použití GUI by měla být
použita knihovna SWT\cite{swt}.

\subsubsection*{Funkčnost na operačních systémech Windows a Linux}

Tento požadavek opět souvisí s tím, že existující aplikace podporují právě tyto
operační systémy. Navíc jsou systémy Windows a Linux nejsnáze dostupné pro
testovací účely. 

Ačkoliv je Java multiplatformní jazyk, možnosti týkající se zístkávání informací
specifických pro konkrétní operační systém jsou značně omezené. Pokud chceme
vázat licenci na konkrétní hardware, je zapotřebí mnohem těsnější vazby s
operačním systémem, než jakou Java umožňuje. Z tohoto důvodu bude nutné část
funkcionality přizpůsobit konkrétnímu operačnímu systému a tuto část
implementovat pro všechny podporované platformy. Jako cílové byly zvoleny
operační systémy Windows a Linux, mělo by ale být možné v případě potřeby snadno
rozšířit podporu i pro další operační systémy.


\subsubsection*{Snadná integrovatelnost do již existujících aplikací}
 
K rozhodnutí o přidání funkčnosti pro licencování do aplikace může dojít až v
pozdější fázi vývoje, případně může být požadavek licencování přidat do již
hotového produktu. Z tohoto důvodu by měla být integrace co nejjednodušší.
Modulární architektura platformy Eclipse RCP toto v případě integrace do
desktopové aplikace usnadní. Stejný požadavek ale můžemem mít také na serverové
straně. Pokud již máme existující databázi uživatelů, mohli bychom ji chtít
využít pro vystavení licencí.
  
\subsubsection*{Centrální správa licencí}

Aby bylo možné mít přehled o všech vydaných licencích, je nutné licence
spravovat centrálně.

\subsubsection*{Webové GUI pro snadnějš správu licencí}

Protože o vydávání licencí se běžně stará obchodní oddělení, je potřeba
vytvořit grafické uživatelské rozhraní, které umožní uživatelům spravovat vydané
licence. Z hlediska snadnosti použití a platformní nezávislosti se jako
nejvhodnější jeví webové rozhraní.


\subsection{Funkční požadavky}

\subsubsection*{Vlastnosti licence}

Zde jsou uvedeny vlastnosti, které bude mít vydaná licence.

\begin{itemize}
  \item Časové omezení – počátek nebo konec platnosti licence je možné časově
  omezit
  \item Vazba na hardware – platnost licence je možné vázat na konkrétní
  hardware
  \item Licencované vlastnosti - licence bude obsahovat seznam vlastností, které
  jsou v licenci zahrnuty. To umožní aplikaci omezit funkcionalitu jen na
  vlastnosti uvedené v licenci.
\end{itemize}

\subsubsection*{Vlastnosti klientské části}

Tyto vlastnosti bude mít klientská část licenčního mechanismu, která se bude
zabudovávat do aplikace.

\begin{itemize}
  \item Online kontrola platnosti – platnost licence bude možné ověřit proti
  licenčnímu serveru.
  \item Offlice kontrola platnosti – platnost údajů uvedených v licenci bude
  možné ověřit bez nutnosti připojení k centrálnímu serveru. Jedná se především
  o kontrolu dat počátku a konce platnosti a kontroly vazby na hardware.
\end{itemize}

\subsubsection*{Vlastnosti serverové části}

\begin{itemize}
  \item Zobrazení přehledu vystavených licencí
  \item Vytvoření nové licence
  \item Zneplatnění licence
  \item Ověření platnosti licence
\end{itemize}

\section{Architektura}

Jako 

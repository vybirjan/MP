\chapter{Ochrana aplikace}

Cílem této kapitoly je prozkoumat problematiku zabezpečení aplikací napsaných v
jazyce Java proti modifikacím a navrhnout možné způsoby, jak se neautorizovaným
úpravám zabránit.

\section{Bezpečnostní rizika}

Pokud chceme v aplikaci použít licenční mechanismus, který omezí funkčnost
aplikace na základě vystavené licence, je potřeba aplikaci zabezpečit proti
neautorizovaným úpravám. Aby byl licenční mechanismus účinný, nemělo by být
možné ho jendoduše obejít nebo úplně odstranit z aplikace.

V této části jsou popsány hlavní problémy se zabezpečením aplikací napsaných v
Javě proti modifikacím.

\subsection{Čitelnost bytecode}

Aplikace napsané v Javě jsou, na rozdíl od aplikací napsaných například v
jazycích C/C++, multiplatformní. To znamená, že aplikaci napsanou a
zkompilovanou na Windows je možné spustit bez dalšího kompilování i na jiných
operačních systémech. Toho je dosaženo díky použití bytecode. Aplikace napsané v
Javě nejsou kompilovány přímo do strojového kódu pro danou platformu, ale do
tzv. bytecode, který je interpretován virtuálním strojem Javy. Při kompilaci je
pro každou třídu vytvořen samostatný soubor s příponou .class, který kromě
instrukcí pro virtuální stroj javy obsahuje i popis všech metod a atributů dané
třídy.

Soubory obsahující definice tříd mají přesně specifikovanou a dobře
zdokumentovanou strukturu a jsou uloženy v čitelné podobě. Navíc bytecode
obsahuje v provonání se strojovým kódem instrukce na mnohem vyšší úrovni
abstrakce. Bytecode obsahuje například i instrukce pro přístup k atributům tříd
nebo volání metod. Díky tomu je možné dekompilací z .class souborů poměrně
snadno získat zpět zdrojový kód, který je navíc velmi podobný původnímu
zdrojovému kódu, jehož kompilací byl .class soubor vytvořen.

Ukázka dekompilace kódu je TODO


Díky možnosti dekompilace a úpravy již zkompilovaných souborů může útočník
snadno z aplikace odstranit volání licenční knihovny a tím docílit neomezeného
využívání licencované aplikace.

\subsection{Linkování knihoven}



\chapter{Ochrana aplikace}

Cílem této kapitoly je prozkoumat problematiku zabezpečení aplikací napsaných v
jazyce Java proti modifikacím a navrhnout možné způsoby, jak se neautorizovaným
úpravám zabránit.

\section{Bezpečnostní rizika}

Pokud chceme v aplikaci použít licenční mechanismus, který omezí funkčnost
aplikace na základě vystavené licence, je potřeba aplikaci zabezpečit proti
neautorizovaným úpravám. Aby byl licenční mechanismus účinný, nemělo by být
možné ho jendoduše obejít nebo úplně odstranit z aplikace.

V této části jsou popsány hlavní problémy se zabezpečením aplikací napsaných v
Javě proti modifikacím.

\subsection{Čitelnost bytecode}

Aplikace napsané v Javě jsou, na rozdíl od aplikací napsaných například v
jazycích C/C++, multiplatformní. To znamená, že aplikaci napsanou a
zkompilovanou na Windows je možné spustit bez dalšího kompilování i na jiných
operačních systémech. Toho je dosaženo díky použití bytecode. Aplikace napsané v
Javě nejsou kompilovány přímo do strojového kódu pro danou platformu, ale do
tzv. bytecode, který je interpretován virtuálním strojem Javy. Při kompilaci je
pro každou třídu vytvořen samostatný soubor s příponou .class, který kromě
instrukcí pro virtuální stroj javy obsahuje i popis všech metod a atributů dané
třídy.

Soubory obsahující definice tříd mají přesně specifikovanou a dobře
zdokumentovanou strukturu a jsou uloženy v čitelné podobě. Navíc bytecode
obsahuje v provonání se strojovým kódem instrukce na mnohem vyšší úrovni
abstrakce. Bytecode obsahuje například i instrukce pro přístup k atributům tříd
nebo volání metod. Díky tomu je možné dekompilací z .class souborů poměrně
snadno získat zpět zdrojový kód, který je navíc velmi podobný původnímu
zdrojovému kódu, jehož kompilací byl .class soubor vytvořen.

Ukázku toho, jak může vypadat dekompilovaný kód, je možné vidět na ukázce
\ref{lst:decompiled-src}. Původní zdrojový kód třídy Decompilation je na ukázce
\ref{lst:original-src}.

\cleardoublepage
\lstset{caption={Originální zdrojový kód třídy},label=lst:original-src}

\begin{lstlisting}
public class Decompilation {

	public enum TestEnum {
		A, B, C
	}

	private static int integerVariable = 0;
	static {
		integerVariable = 5 * (22 + 16);
	}

	private boolean flag = false;

	public TestEnum getValue() {
		return TestEnum.values()[(int) (System.currentTimeMillis() % 3)];
	}

	public String testMethod() {
		String localunusedVar = "test" + " integerVariable= " + integerVariable + ".";
		float variable = Float.MAX_VALUE;
		while (flag) {
			variable -= Float.MAX_VALUE / 10;
			flag = variable > 0;
		}
		switch (getValue()) {
			case A:
				System.out.print("A");
				break;
		}
		for (int var = 0;; var++) {
			if (var > 100) {
				break;
			}
			System.out.println(var);
		}
		return testObj.toString();
	}

	private final Object testObj = new Object() {
		@Override
		public String toString() {
			return String.valueOf(integerVariable);
		};
	};
}
}
\end{lstlisting}
\newpage

\lstset{caption={Dekompilovaný zdrojový kód třídy},label=lst:decompiled-src}

\begin{lstlisting}
public class Decompilation {
	private static int integerVariable = 0;
	private boolean flag = false;
	private final Object testObj = new Object() {
		public String toString() {
			return String.valueOf(Decompilation.integerVariable);
		}
	};
	static {
		integerVariable = 190;
	}
	public TestEnum getValue() {
		return TestEnum.values()[(int) (System.currentTimeMillis() % 3L)];
	}
	public String testMethod() {
		new StringBuilder("test integerVariable= ").append(integerVariable).append(".").toString();
		float f = 3.4028235E+38F;
		while (this.flag) {
			f -= 3.402824E+037F;
			this.flag = (f > 0.0F);
		}
		switch ($SWITCH_TABLE$test$Decompilation$TestEnum()[getValue().ordinal()]) {
			case 1:
				System.out.print("A");
				break;
		}
		for (int i = 0; i <= 100; i++)
			System.out.println(i);
		return this.testObj.toString();
	}

	public static void main(String[] paramArrayOfString) {
		Decompilation localDecompilation = new Decompilation();
		localDecompilation.testMethod();
	}
	public static enum TestEnum {
		A, B, C;
	}
}
\end{lstlisting}

\newpage

Díky možnosti dekompilace a úpravy již zkompilovaných souborů může útočník
snadno z aplikace odstranit volání licenční knihovny a tím docílit neomezeného
využívání licencované aplikace.

\subsection{Linkování knihoven}

Při kompilaci aplikace je nutné k výslednému programu připojit také používané
knihovny. Tomuto kroku se říká linkování a rozlišujeme dva základní způsoby –
statické a dynamické liknování.

V případě statického liknování jsou všechny závislosti na knihovy vyřešeny během
kompilace a jsou zabaleny společne s kódem aplikace do výsledného spustitelného
souboru. Tento způsob je často používán u jazyků kompilovaných do strojového
kódu jako je například C nebo C++.

U dynamického linkování jsou knihovny umístěny mimo výsledný zkompilovaný
spustitelný soubor a můžou být k programu připojeny až ve chvíli, kdy je volána
některá jejich metoda. 

Dynamické linkování je použito v Javě pro načítání tříd. Načtení třídy se
provede za běhu programu až ve chvíli, kdy je třída poprvé použita. Pro
identifikaci tříd se používá celé jméno třídy včetně balíčku, ve kterém je třída
umístěna. Ve zkompilovaném programu je tedy uložen pouze název odkazované třídy,
případně názvy volaných metod. To může představovat bezpečnostní riziko pro
aplikaci.

Pokud bychom měli například třídu \textbf{com.test.LicenseService} a aplikace by
pomocí této třídy prováděla ověřování licence, útočníkovi stačí tuto třídu
nahradit vlastní implementací se stejným jménem a signaturou. Aplikace by při
prvním použití načetla útočníkovu třídu a použila ji pro ověření licence.

U aplikací postavených na platformě \gls{OSGi} je tato vlastnost Javy ještě
problematičtější. Pokud by byla část starající se o licencování přidána do
aplikace jako plugin, dá se snadno odstranit a nahradit upravenou verzí. Pluginy
navíc obsahují popis všech sdílených balíků a poskytovaných služeb, takže jejich
nahrazení je ještě o něco snazší. Pokud by byla třída navíc publikována jako
\gls{OSGi} služba, stačilo by útočníkovi vytvořit vlastní plugin, který by
zaregistroval stejnou službu s větší prioritou.


\section{Možné způsoby ochrany}

\subsection{Obfuskace}

Obfuskace je úprava kódu aplikace takovým způsobem, aby se tím zakryl jeho
původní účel. Obfuskovaný kód aplikace vykonává stejnou činnost jako kód před
obfuskací, je ovšem mnohem obtížnější z něj odahlit, co přesně má kód vykonávat.
Pro Javu existuje celá řada různých nástrojů, které umožňují obfuskovat
aplikace, používá se buď obfuskace zdrojových kódu nebo obfuskace zkompilovaných
tříd. Obfuskace na úrovni zdrojového kódu je vhodná spíše v případě, že zdrojový
kód je dostupný cizím osobám a chceme zamezit tomu, aby mohli kód použít k
dalšímu vývoji. Obfuskace bytecode znesnadňuje dekompilaci obfuskovaných tříd.
Výhodou obfuskace na úrovni bytecode je fakt, že pravidla pro bytecode jsou méně
restriktivní než které uplatňuje překladač Javy. Tím můžeme zkompilovaný program
upravit tak, že z něj nepůjde zpětně zrekonstruovat platný zdrojový kód v Javě.

Při obfuskování programů se používají různé techniky. Nejčastěji se
přejmenovávají názvy balíčků, tříd, metod a proměnných tak, aby z nich nešlo
jednoduše poznat, jakou funkcni zastávají. Často se také do kódu přidávají
složité řídící struktury nebo zbytečná volání metod.

Ačkoliv obfuskací je možné zvýšit obtížnost dekompilace aplikace a následného
odstranění licenčního mechanismu, existuje několik důvodů, které mohou bránit
jejímu použití:

\begin{itemize}
  \item Při změně názvů tříd a metod je nutné tyto změny propagovat i do dalších
  částí programu, které s obfuskovanou komponentou pracují. Pokud například
  obfuskujeme \gls{OSGi} plugin, je nutné buď veřejné \gls{API} nechat nedotčené
  nebo provádět obfuskaci nad všemi pluginy najednou. Obfuskování veřejného
  \gls{API} pluginů ovšem značně znesnadňuje další rozšíření aplikace.
  \item Obfuskováním může docházet ke snižování výkonu aplikace. Obfuskací se
  mohou přidávat zbytečná volání metod nebo složité programové konstrukce, které
  jsou složité na vyhodnocení.
  \item Protože obfuskace změní kód aplikace tak, že neodpovídá zdrojovému kódu,
  může být hledání chyb v obfuskovaném programu komplikované.
  \item Obfuskace nepřináší žádnou bezpečnost navíc. Pomocí obfuskace se sice
  získání původního kódu případně nalezení míst se zabezpečením znesnadní,
  aplikaci ale bude stále možné libovolně dekompilovat a upravit.
\end{itemize}


\subsection{Kontrola integrity aplikace}

Jako další z možností ochrany aplikace proti modifikacím se nabízí použítí
zabudované kontroly integrity. Princip spočívá v tom, že aplikace má v sobě
uložené informace o podobě jednotlivých částí (například v podobě
kontrolních součtů nebo otisků). Při startu aplikace nebo za běhu se poté
provede kontrola proti těmto informacím. V případě, že je zjištěn rozdíl, se
aplikace ukončí.

Účinnost této techniky je možné ještě zvýšit tím, že se kontrola integrity
aplikace zduplikuje na větší počet míst v kódu. Pokud by například licencovací
funkcionalita byla aplikaci poskytnuta v podobě \gls{OSGi} služby, před volání
této služby by se umístil kód, který by ověřil, že služba byla poskytnuta ze
správnéhu pluginu a že jeho otisk se nezměnil. Opakované ruční vkládání stejného
kódu na velké množství míst ale ztěžuje údržbu aplikace a případné změny by bylo
nutné aplikovat opakovaně, čímž se zvyžuje riziko zanesení chyb do aplikace.
Proto je vhodné pro tento účel použít specializované nástroje pro manipulaci s
bytecode, například ASM\cite{asm}.

Nevýhodou zabudování kontroly integrity do aplikace je hned několik:
\begin{itemize}
  \item Je nutné udržovat otisky kontrolovaných částí aktuální. Pokud se upraví
  část aplikace, je nutné příslušný otisk přepočítat a nahradit ho na všech
  místech, kde je použit.
  \item Použití této techniky nijak nezabezpečí samotný kód. Ten lze stále
  dekompilovat a kontrolu otisků lze odstranit. Případně je možné jen upravit
  uložené otisky aby odpovídali upravenému kódu.
  \item Přepočítávání otisků může být náročné na výkon.
\end{itemize}


\subsection{Šifrování bytecode}

Šifrování bytecode je účinná metoda, která zamezuje dekompilaci a modifikaci
aplikace. Zkompilované .class soubory jsou zašifrovány pomocí symetrické šifry.
Jejich rozšifrování se provádí až za běhu před načtením zašifrované třídy pomocí
uloženého klíče. Díky tomu není možné aplikaci jednoduše dekompilovat. Protože
jsou .class soubory šifrované, nelze je přečíst pomocí dekompilátoru.

Dešiforvání tříd při načtení je možné provést různými způsoby. U běžných
aplikací napsaných v Javě je možné pro načítání šifrovaných tříd použít vlastní
classloader. U aplikací postavených na platformě \gsl{OSGi} je situace ještě
jednodušší. Platforma \gls{OSGi} umožňuje aplikacím definovat tzv. Adaptor
Hooks. Jedná se o třídy, které rozšiřují funkcionalitu platformy a je možné je
do aplikace přidat pomocí jednoduché konfigurace. Pro účely dešifrování bytecode
se nejlépe hodí ClassLoadingHook. Ten umožňuje upravit bytecode třídy ještě před
tím, než se načte pomocí classloaderu.

Největším problémem této metody je potřeba mít uložený šifrovací klíč v
aplikaci v čitelné podobě, aby bylo možné provést dešifrování tříd. Z toho
vyplývá, že i když jsou třídy šifrované a nelze je dekompilovat, je možné z
aplikace získat klíč a dešifrovat je pomocí něj. Tento nedostatek je možné
částečně řešit zapojením licencovacího mechanismu. Šiforvací klíč může
být do aplikace dodán až společne s platnou licencí. Navíc je možné šifovat
různými klíči různé části aplikace a v licenci poslat jen klíče k částem
programu, které jsou v licenci zahrnuty. Tím zajistíme, že bez platné licence
nepůjde aplikace používat a ani nebude možné aplikaci dekompilovat a ochranu z
ní odstranit.

Ani šifrování tříd neposkytuje neproniknutlenou ochranu. Bytecode tříd musí být
za běhu dostupný v dešifrované podobě, aby jej bylo možné spouštět. Útočník tak
může do aplikace přidat kód, který si pomocí deěifrovacího classloaderu načte
bytecode libovolné třídy a poté si je může v nezašifrované podobě uložit na
disk. Navíc kód aplikace sloužící k dešiforvání tříd musí být dostupný v
nezašiforvané podobě, aby jej bylo možné načíst a použít pro dešifrování zbylých
tříd. Dekompilací této části může útočník získa dostatek informací o použitém
šiforvacím algoritmu a případně i šifrovací klíče potřebné pro dešiforvání
zbytku aplikace.

\subsection{Kompilace do nativního kódu}

Kompilací aplikace do nativního kódu lze dosáhnout dobrého zabezpečení aplikace.
Při této metodě je aplikace překládána místo do bytecode přímo do strojového
kódu pro danou platformu. Během překladu se provádí řada úprav a optimalizací a
díky tomu je získání původního zdrojového kódu téměř nemožné. Pro kompilaci
programů napsaných v Javě existuje několik nástrojů:

\begin{itemize}
  \item GNU Compiler for Java\cite{gcj} - Jedná se o open-source kompilátor,
  který umožňuje aplikace napsané v Javě kompilovat jak do bytecode, tak i do
  strojového kódu pro různé platformy, podporuje i překlad bytecode do
  strojového kódu. Nevýhodou GCJ je chybějící podpora pro \gls{GUI} a některé
  další součásti standardní knihovny Javy, podporovaná je pouze Java ve
  verzi 1.4 a částečně 1.5.
  \item Excelsior JET\cite{jet} - JET je sada nástrojů a běhové prostředí pro
  aplikace napsané v Javě. Umožňuje kompilovat Javovské aplikace do strojového
  kódu a spouštět je ve vlastním běhovém prostředí, které je kompatibilní s
  Javou verze 1.6. Protože se jendá o komerční produkt, jako největší nevýhoda
  může být cena, která se pohybuje řádově v tisících euro za platformu pro
  komerční použití. Pro nekomerční použití je možné získat licenci zdarma.
\end{itemize}

Nevýhodou kompilování Javovských aplikací do strojového kódu je především ztráta
přenositelnosti mezi platformami. Pro každou podporovanou platformu je nutné
zkompilovat aplikaci samostatně. Výhodou může být kromě zmíněné ochrany proti
dekompilaci také rychlejší start aplikace, rychlost aplikace za běhu se ale při
použití moderních virtuálních strojů Javy příliš neliší.


\subsection{Hardwarová ochrana}

Kromě softwarových řešení existují i čistě hardwarová řešení pro ochranu
Javovských aplikací. Příkladem může být produkt Validy
SoftNaos\cite{validy-softnaos} v podobě přídavného USB zařízení. Ochrana
poskytovaná tímto zařízením spočívá v tom, že se část kódu apliakce po přeložení
zašifruje. Takto zašifrovaný kód je poté možné spustit pouze pomocí připojeného
USB zařízení. Takto chráněný kód není možné dekompilovat a upravit a to ani za
běhu aplikace, díky čemuž může být tento způsob ochrany jako jediný ze zmíněných
považován za bezpečný.

Nevýhodou při použití hardwarové ochrany je především nutnost společně s
aplikací distribuovat i hardwarová zařízení pro spouštění šifrovaného kódu,
jejichž cena může často i přesáhnout cenu samotné aplikace. Problematický může
být také výkon takto chráněného kódu, vykonávání šifrovaného kódu může být i
$40000\times$ pomalejší než u nešifrovaného kódu\cite{validy-softnaos-review}.

\section{Závěr}

Zabezpečení aplikací napsaných v jazyce Java je díky snadné dekompilaci bytecode
problematické. Kromě použití speciálního hardwarového zařízení pro dešifrování
bytecode neexistuje spolehlivá metoda, pomocí níž by šlo zabránit dekompilaci a
modifikaci kódu aplikace. %TODO

Pro implementaci v rámci této práce byla zvolena metoda šifrování kompilovaných
tříd s distribucí šifrovacího klíče pomocí licenčního mechanismu. Pro zvýšení
úrovně bezpečnosti je vhodné tuto metodu kombinovat například s obfuskováním
kódu aplikace nebo s kompilováním části starající se o dešifrování tříd do
nativního kódu.

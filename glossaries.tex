\newglossaryentry{gls-gui}
{
	name=Graphic User Interface,
	description={je uživatelské rozhraní, které umožňuje ovládat počítač pomocí
	interaktivních grafických ovládacích prvků.}
}

\newglossaryentry{gls-sdk}
{
	name=Software Development Kit,
	description={je sada softwarových nástrojů umožňující tvorbu aplikací pro
	určitý softwarový balík, platformu, operační systém nebo hardware}
}

\newglossaryentry{gls-eclipse-rcp}
{
	name=Eclipse Rich Clilent Platform,
	description={je součást projektu Eclipse zaměřující se na tvorbu bohatých
	desktopových aplikací} 
}

\newglossaryentry{gls-osgi}
{
	name=Open Services Gateway initiative framework,
	description={je specifikace dynamického modulárního systému pro jazyk Java} 
}

\newglossaryentry{gls-mac}
{
	name=MAC adresa,
	description={je } 
}

\newglossaryentry{gls-json}
{
	name=JavaScript Object Notation,
	description={je textový formát pro přenos dat odvozený od jazyja JavaScript
	umožňující jendoduchý zápis datových struktur} 
}

\newglossaryentry{gls-xml}
{
	name=Extensible Markup Language,
	description={je značkovací jazyk, který definuje pravidla pro zápis dokumentů
	v textové podobě vhdoné pro strojové čtení} 
}

\newglossaryentry{gls-http}
{
	name=Hypertext Transfer Protocol,
	description={je jeden ze základních protokolů používaných ke komunikaci na
	internetu.} 
}

\newglossaryentry{gls-jni}
{
	name=Java Native Interface,
	description={je rozhraní umožňující propojit kód běžící ve virtuálním stroji
	Javy s nativními programy a knihovnami napsanými v jiných jazycích.}
}

\newglossaryentry{gls-jna}
{
	name=Java Native Access,
	description={je knihovna, která usnadňuje propojení kódu běžícího ve
	virtuálním stroji Javy s nativními programy a knihovnami. Narozdíl od
	JNI není při propojení aplikací pomocí JNA potřeba speciálních hlavičkových
	souborů v jazyce C. }
}

\newglossaryentry{gls-swt}
{
	name=Standard Widget Toolkit,
	description={je knihovna pro vytváření grafického uživatelského rozhraní pro
	jazyk Java. Narozdíl od knihovny Swing, které je součástí standardní
	distribuce Javy, knihovna SWT využívá pro vykreslování grafických prvků volání
	nativních metod operačního systému. Díky tomu má grafické uživatelské rozhraní
	vytvořené pomocí knihovny SWT stejný vzhled jako nativní aplikace.}
}


\newacronym{GUI}{GUI}{\glslink{gls-gui}{Graphic User Interface}}
\newacronym{SDK}{SDK}{\glslink{gls-sdk}{Software Development Kit}}
\newacronym{OSGi}{OSGi}{\glslink{gls-osgi}{Open Services Gateway initiative framework}}
\newacronym{MAC}{MAC}{\glslink{gls-mac}{Media Access Control}}
\newacronym{JSON}{JSON}{\glslink{gls-json}{JavaScript Object Notation}}
\newacronym{XML}{XML}{\glslink{gls-xml}{Extensible Markup Language}}
\newacronym{DAO}{DAO}{Data Access Object}
\newacronym{REST}{REST}{Representational State Transfer}
\newacronym{HTTP}{HTTP}{\glslink{gls-http}{Hypertext Transfer Protocol}}
\newacronym{HTTPS}{HTTPS}{\glslink{gls-http}{Hypertext Transfer Protocol
Secure}}
\newacronym{URL}{URL}{Unified Resource Locator}
\newacronym{JNI}{JNI}{\glslink{gls-jni}{Java Native Interface}}
\newacronym{JNA}{JNA}{\glslink{gls-jna}{Java Native Access}}
\newacronym{SWT}{SWT}{\glslink{gls-swt}{Standard Widget Toolkit}}
\newacronym{API}{API}{Application programming interface}